\begin{tikzpicture}[scale=1.0]

    % Define coordinates for the points based on approximate proportions
    % Point B at the origin
    \coordinate (B) at (0,0);
    % Point C on the x-axis
    \coordinate (C) at (3.5,0);
    % Point D extended further right on the x-axis
    \coordinate (D) at (7,0);
    % Point A at the top, creating roughly an isosceles triangle
    \coordinate (A) at (1.75, 3.5);

    % Draw the horizontal line segment BD
    \draw[thick] (B) -- (D);

    % Draw the triangle sides AB and AC
    \draw[thick] (B) -- (A) -- (C);

    % Draw the angle arc at Vertex A
    % The angles are approximated based on the coordinates of A, B, and C
    % (Vector AB is approx 243 degrees, Vector AC is approx 297 degrees)
    \draw[semithick] (A) + (243:0.6cm) arc (243:297:0.6cm);

    % Label the angle value inside the triangle
    \node at (1.75, 2.7) {$40^{\circ}$};

    % Place labels for the points exactly as shown in the image
    \node[above] at (A) {A};
    \node[left] at (B) {B};
    \node[below] at (C) {C};
    \node[right] at (D) {D};

\end{tikzpicture}