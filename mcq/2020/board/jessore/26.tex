\documentclass[tikz, border=10pt]{standalone}
\usepackage{fontspec}
\usetikzlibrary{patterns}

% Set the Bengali font for Overleaf (XeLaTeX or LuaLaTeX)
\newfontfamily\bengalifont[Script=Bengali]{Noto Sans Bengali}

\begin{document}

\begin{tikzpicture}[scale=1]

% 1. Define dimensions
% Using 10 and 6 to match your labels exactly
\def\rectwidth{10}
\def\rectheight{6}

% 2. Draw the shaded rectangle with horizontal pattern
\draw[ultra thick, pattern={horizontal lines}, pattern color=gray!60] (0,0) rectangle (\rectwidth, \rectheight);

% 3. Draw the white circle in the center
% Radius is 3 (half of height 6) so it touches the top and bottom edges
\fill[white] (\rectwidth/2, \rectheight/2) circle (\rectheight/2);
\draw[ultra thick] (\rectwidth/2, \rectheight/2) circle (\rectheight/2);

% 4. Center point of the circle
\filldraw (\rectwidth/2, \rectheight/2) circle (2.5pt);

% 5. Right-angle symbol at the bottom-left corner
\draw[thick] (0, 0.4) -- (0.4, 0.4) -- (0.4, 0);

% 6. Labels in Bengali (সে.মি.)
% Bottom measurement (১০ সে.মি.)
\node[below, font=\Large, yshift=-8pt] at (\rectwidth/2, 0) {{\bengalifont ১০ সে.মি.}};

% Right side measurement (৬ সে.মি.)
\node[right, font=\Large, xshift=8pt] at (\rectwidth, \rectheight/2) {{\bengalifont ৬ সে.মি.}};

\end{tikzpicture}

\end{document}
