\documentclass{standalone}
\usepackage{tikz}
\usepackage{fontspec}
\usetikzlibrary{matrix}

% Set up Bengali font (use any Bengali font installed on your system)
\newfontfamily\bengalifont{Noto Sans Bengali}[Script=Bengali]

\begin{document}
\begin{tikzpicture}

\matrix (table) [
  matrix of nodes,
  nodes={draw, minimum height=0.8cm, text depth=0.25ex, text height=2ex, anchor=center},
  column 1/.style={nodes={minimum width=2.4cm}},
  column 2/.style={nodes={minimum width=1.1cm}},
  column 3/.style={nodes={minimum width=1.1cm}},
  column 4/.style={nodes={minimum width=1.1cm}},
  column 5/.style={nodes={minimum width=1.1cm}},
  column 6/.style={nodes={minimum width=1.1cm}},
  column 7/.style={nodes={minimum width=1.1cm}},
  row sep=-\pgflinewidth,
  column sep=-\pgflinewidth
]
{
  {\bengalifont শ্রেণি ব্যাপ্তি} & 40--44 & 45--49 & 50--54 & 55--59 & 60--64 & 65--69 \\
  {\bengalifont গণসংখ্যা} & 6 & 10 & 16 & 20 & 13 & 5 \\
};

\end{tikzpicture}
\end{document}
