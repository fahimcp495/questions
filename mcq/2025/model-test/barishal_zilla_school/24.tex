\documentclass[tikz, border=5pt]{standalone}
\usepackage{fontspec}
\setmainfont{Noto Sans Bengali}

\begin{document}
\begin{tikzpicture}[scale=1]

    % Define the center of the circles
    \coordinate (O) at (0,0);

    % Fill the sector region AOB first so borders draw over it
    % Using gray!50 to represent the shaded (গাঢ়) region clearly
    \fill[gray!50] (O) -- (-120:2.4) arc (-120:-60:2.4) -- cycle;

    % Draw the inner circle
    \draw[thick] (O) circle (2.4);

    % Draw the outer circle 
    \draw[thick] (O) circle (3.2);

    % Define the points A and B on the inner circle
    \coordinate (A) at (-120:2.4);
    \coordinate (B) at (-60:2.4);

    % Draw the boundary radii for the shaded sector
    \draw[thick] (O) -- (A);
    \draw[thick] (O) -- (B);

    % Add the labels for the points exactly as shown in the image
    \node at (0, 0.4) {$O$};
    \node at (-125:2.8) {$A$};
    \node at (-55:2.8) {$B$};
    
    % Label C is situated in the annular gap in the top-left quadrant
    \node at (135:2.8) {$C$};

    % Add the measurement labels using math mode numbers and Bengali text
    % 6 cm is placed inside the sector and shifted further to the right
    \node[right] at (1.0, -1.2) {$6$ সেমি}; 
    
    % 2 cm is placed in the gap and shifted a bit to the right using xshift
    \node[right, xshift=0.3cm] at (25:2.8) {$2$ সেমি};  

\end{tikzpicture}
\end{document}