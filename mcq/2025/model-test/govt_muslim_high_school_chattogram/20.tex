\begin{tikzpicture}[scale=1]

  % Define the center of the circle and its radius
  \coordinate (O) at (0,0);
  \def\R{1.2}

  % Draw the circle and fill its center point
  \draw[thick] (O) circle (\R);
  \fill (O) circle (1.5pt);

  % Define the external point M. We use the coordinates from the user's final good derivation block.
  \coordinate (M) at (-3.5, 0.2);

  % Define the points of tangency P and Q. (Angles provide approximate 60° total angle for MP and MQ).
  \coordinate (P) at (115:\R);
  \coordinate (Q) at (245:\R);

  % Draw tangent rays from M extending visually past P and Q.
  \draw[thick] (M) -- ($(M)!1.3!(P)$);
  \draw[thick] (M) -- ($(M)!1.3!(Q)$);

  % Draw radius lines from the center O to P and Q
  \draw[thick] (O) -- (P);
  \draw[thick] (O) -- (Q);

  % Draw angle arc at M to denote the 60° angle.
  % We calculate the correct angles between the vectors MP and MQ.
  \draw[thick] (M) let
      \p1 = ($(P)-(M)$),
      \p2 = ($(Q)-(M)$),
      \n1 = {atan2(\y1,\x1)},
      \n2 = {atan2(\y2,\x2)}
    in
      +(\n1:0.6) arc (\n1:\n2:0.6);

  % Add point labels exactly as in the image.
  \node[left] at (M) {M};
  \node[above] at (P) {P};
  \node[below] at (Q) {Q};
  \node[right] at (O) {O};

  % Add the 60° angle label inside the arc.
  \node at (-2.5, 0.2) {$60^\circ$};

\end{tikzpicture}