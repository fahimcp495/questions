\documentclass[tikz, border=10pt]{standalone}
\usepackage{tikz}
\usetikzlibrary{calc, decorations.pathreplacing}
\pgfmathsetmacro{\angleDC}{atan2(3.5, 4)}

\begin{document}
\begin{tikzpicture}

% Define points
\coordinate (B) at (0, 0);
\coordinate (A) at (0, 5);
\coordinate (C) at (0, 3.5);
\coordinate (D) at (-4, 0);

% Draw vertical line AB
\draw[ultra thick] (A) -- (B);

% Draw line DB
\draw[ultra thick] (D) -- (B);

% Draw line DC
\draw[ultra thick] (D) -- (C);

% Right angle mark at B
\def\sq{0.35}
\draw[thick] (B) ++(-\sq, 0) -- ++(0, \sq) -- ++(\sq, 0);

% Arc at C (angle DCB) - no label
\draw[thick] ($(C)+(270:0.5)$) arc (270:221:0.5);
\node[right, font=\large] at ($(C)+(-130:1.2)$) {$60^{\circ}$};

% Double tick mark on AC
\coordinate (midAC) at ($(A)!0.5!(C)$);
\draw[thick] ($(midAC)+(-0.2, 0.08)$) -- ++(0.4, 0);
\draw[thick] ($(midAC)+(-0.2, -0.08)$) -- ++(0.4, 0);

% Double tick mark on DC (perpendicular to DC)
\coordinate (midDC) at ($(D)!0.5!(C)$);
% First tick
\draw[thick] ($(midDC)+(\angleDC:0.08)$) ++(\angleDC+90:0.2) -- ++(\angleDC-90:0.4);
% Second tick
\draw[thick] ($(midDC)+(\angleDC:-0.08)$) ++(\angleDC+90:0.2) -- ++(\angleDC-90:0.4);


% Labels
\node[above] at (A) {\textbf{\Large A}};
\node[below right] at (B) {\textbf{\Large B}};
\node[right] at ($(C)+(0.15,0)$) {\textbf{\Large C}};
\node[below left] at (D) {\textbf{\Large D}};

% Brace on right side showing 10 m
\draw[decorate, decoration={brace, amplitude=10pt, mirror}, thick] 
    (0.5, 0) -- (0.5, 3.2);
\node[right, font=\Large] at (1.2, 1.5) {10 m};

\end{tikzpicture}
\end{document}